% Reference Card for GNU Emacs

% Copyright (C) 1987, 1993, 1996--1997, 2001--2020 Free Software
% Foundation, Inc.
% Copyright (C) 2020 Masahiro Nakamura

% Author: Stephen Gildea <stepheng+emacs@gildea.com>
% Japanese translation: Masahiro Nakamura <tsuucat@icloud.com>

% This document is free software: you can redistribute it and/or modify
% it under the terms of the GNU General Public License as published by
% the Free Software Foundation, either version 3 of the License, or
% (at your option) any later version.

% As a special additional permission, you may distribute reference cards
% printed, or formatted for printing, with the notice "Released under
% the terms of the GNU General Public License version 3 or later"
% instead of the usual distributed-under-the-GNU-GPL notice, and without
% a copy of the GPL itself.

% This document is distributed in the hope that it will be useful,
% but WITHOUT ANY WARRANTY; without even the implied warranty of
% MERCHANTABILITY or FITNESS FOR A PARTICULAR PURPOSE.  See the
% GNU General Public License for more details.

% You should have received a copy of the GNU General Public License
% along with GNU Emacs.  If not, see <https://www.gnu.org/licenses/>.


% This file is intended to be processed by plain TeX (LuaTeX).
%
% The final reference card has six columns, three on each side.
% This file can be used to produce it in any of three ways:
% 1 column per page
%    produces six separate pages, each of which needs to be reduced to 80%.
%    This gives the best resolution.
% 2 columns per page
%    produces three already-reduced pages.
%    You will still need to cut and paste.
% 3 columns per page
%    produces two pages which must be printed sideways to make a
%    ready-to-use 8.5 x 11 inch reference card.
%    For this you need a dvi device driver that can print sideways.
% Which mode to use is controlled by setting \columnsperpage.
%
% To compile and print this document:
% luatex ja-refcard.tex
%
% Thanks to Paul Rubin, Bob Chassell, Len Tower, and Richard Mlynarik
% for their many good ideas.

%**start of header
\newcount\columnsperpage
\newcount\letterpaper

% This file can be printed with 1, 2, or 3 columns per page.
% Specify how many you want here.
\columnsperpage=3

% Set letterpaper to 0 for A4 paper, 1 for letter (US) paper.  Useful
% only when columnsperpage is 2 or 3.
\letterpaper=1

% PDF output layout.  0 for A4, 1 for letter (US), a `l' is added for
% a landscape layout.
\input luatex85.sty
\input pdflayout.sty
\pdflayout=(1l)

% Nothing else needs to be changed below this line.

\input emacsver.tex

\input luatexja.sty

\def\shortcopyrightnotice{\vskip 1ex plus 2 fill
  \centerline{\small \copyright\ \year\ Free Software Foundation, Inc.
  Permissions on back.}}

\def\copyrightnotice{
\vskip 1ex plus 2 fill\begingroup\small
\centerline{原著}
\centerline{Copyright \copyright\ \year\ Free Software Foundation, Inc.}
\centerline{For GNU Emacs version \versionemacs}
\centerline{Designed by Stephen Gildea}

Released under the terms of the GNU General Public License version 3 or later.

\centerline{日本語訳}
\centerline{Copyright \copyright\ \year\ Masahiro Nakamura}

より詳細なEmacsのドキュメントについては、配布されているEmacsまたは\break
{\tt https://www.gnu.org/software/emacs}を確認してください。
このカードの\TeX{}ソースについては、{\tt https://github.com/tsuu32/emacs-refcard-ja}を確認してください。
\endgroup}

% make \bye not \outer so that the \def\bye in the \else clause below
% can be scanned without complaint.
\def\bye{\par\vfill\supereject\end}

\newdimen\intercolumnskip	%horizontal space between columns
\newbox\columna			%boxes to hold columns already built
\newbox\columnb

\def\ncolumns{\the\columnsperpage}

\message{[\ncolumns\space
  column\if 1\ncolumns\else s\fi\space per page]}

\def\scaledmag#1{ scaled \magstep #1}

% This multi-way format was designed by Stephen Gildea October 1986.
% Note that the 1-column format is fontfamily-independent.
\if 1\ncolumns			%one-column format uses normal size
  \hsize 4in
  \vsize 10in
  \voffset -.7in
  \font\titlefont=\fontname\tenbf \scaledmag3
  \jfont\titlefontj={file:HaranoAjiGothic-Medium.otf:jfm=ujis} at 16.6271pt
  \let\alphabetictitlefont\titlefont
  \gdef\titlefont{\alphabetictitlefont\titlefontj}
  \font\headingfont=\fontname\tenbf \scaledmag2
  \jfont\headingfontj={file:HaranoAjiGothic-Medium.otf:jfm=ujis} at 13.8559pt
  \let\alphabeticheadingfont\headingfont
  \gdef\headingfont{\alphabeticheadingfont\headingfontj}
  \font\smallfont=\fontname\sevenrm
  \let\smallfontj\sevenmin
  \let\alphabeticsmallfont\smallfont
  \gdef\smallfont{\alphabeticsmallfont\smallfontj}
  \font\smallsy=\fontname\sevensy

  \footline{\hss\folio}
  \def\makefootline{\baselineskip10pt\hsize6.5in\line{\the\footline}}
\else				%2 or 3 columns uses prereduced size
  \hsize 3.2in
  \if 1\the\letterpaper
     \vsize 7.95in
  \else
     \vsize 7.65in
  \fi
  \hoffset -.75in
  \voffset -.745in
  \font\titlefont=cmbx10 \scaledmag2
  \jfont\titlefontj={file:HaranoAjiGothic-Medium.otf:jfm=ujis} at 13.8559pt
  \let\alphabetictitlefont\titlefont
  \gdef\titlefont{\alphabetictitlefont\titlefontj}
  \font\headingfont=cmbx10 \scaledmag1
  \jfont\headingfontj={file:HaranoAjiGothic-Medium.otf:jfm=ujis} at 11.5466pt
  \let\alphabeticheadingfont\headingfont
  \gdef\headingfont{\alphabeticheadingfont\headingfontj}
  \font\smallfont=cmr6
  \jfont\smallfontj={file:HaranoAjiMincho-Medium.otf:jfm=ujis} at 5.7733pt
  \let\alphabeticsmallfont\smallfont
  \gdef\smallfont{\alphabeticsmallfont\smallfontj}
  \font\smallsy=cmsy6
  \font\eightrm=cmr8
  \font\eightbf=cmbx8
  \font\eightit=cmti8
  \font\eighttt=cmtt8
  \font\eightmi=cmmi8
  \font\eightsy=cmsy8
  \jfont\eightmin={file:HaranoAjiMincho-Medium.otf:jfm=ujis} at 7.6977pt
  \jfont\eightgt={file:HaranoAjiGothic-Medium.otf:jfm=ujis} at 7.6977pt
  \textfont0=\eightrm
  \textfont1=\eightmi
  \textfont2=\eightsy
  \def\rm{\eightrm}
  \def\mc{\eightmin}
  \let\alphabeticfont\rm
  \gdef\rm{\alphabeticfont\mc}
  \def\bf{\eightbf\eightgt}
  \def\it{\eightit}
  \def\tt{\eighttt}
  \if 1\the\letterpaper
     \normalbaselineskip=.8\normalbaselineskip
  \else
     \normalbaselineskip=.7\normalbaselineskip
  \fi
  \normallineskip=.8\normallineskip
  \normallineskiplimit=.8\normallineskiplimit
  \normalbaselines\rm		%make definitions take effect

  \if 2\ncolumns
    \let\maxcolumn=b
    \footline{\hss\rm\folio\hss}
    \def\makefootline{\vskip 2in \hsize=6.86in\line{\the\footline}}
  \else \if 3\ncolumns
    \let\maxcolumn=c
    \nopagenumbers
  \else
    \errhelp{You must set \columnsperpage equal to 1, 2, or 3.}
    \errmessage{Illegal number of columns per page}
  \fi\fi

  \intercolumnskip=.46in
  \def\abc{a}
  \output={%			%see The TeXbook page 257
      % This next line is useful when designing the layout.
      %\immediate\write16{Column \folio\abc\space starts with \firstmark}
      \if \maxcolumn\abc \multicolumnformat \global\def\abc{a}
      \else\if a\abc
	\global\setbox\columna\columnbox \global\def\abc{b}
        %% in case we never use \columnb (two-column mode)
        \global\setbox\columnb\hbox to -\intercolumnskip{}
      \else
	\global\setbox\columnb\columnbox \global\def\abc{c}\fi\fi}
  \def\multicolumnformat{\shipout\vbox{\makeheadline
      \hbox{\box\columna\hskip\intercolumnskip
        \box\columnb\hskip\intercolumnskip\columnbox}
      \makefootline}\advancepageno}
  \def\columnbox{\leftline{\pagebody}}

  \def\bye{\par\vfill\supereject
    \if a\abc \else\null\vfill\eject\fi
    \if a\abc \else\null\vfill\eject\fi
    \end}
\fi

% we won't be using math mode much, so redefine some of the characters
% we might want to talk about
\catcode`\^=12
\catcode`\_=12

\chardef\\=`\\
\chardef\{=`\{
\chardef\}=`\}

\hyphenation{mini-buf-fer}

\parindent 0pt
\parskip 1ex plus .5ex minus .5ex

\def\small{\smallfont\textfont2=\smallsy\baselineskip=.8\baselineskip}

% newcolumn - force a new column.  Use sparingly, probably only for
% the first column of a page, which should have a title anyway.
\outer\def\newcolumn{\vfill\eject}

% title - page title.  Argument is title text.
\outer\def\title#1{{\titlefont\centerline{#1}}\vskip 1ex plus .5ex}

% section - new major section.  Argument is section name.
\outer\def\section#1{\par\filbreak
  \vskip 3ex plus 2ex minus 2ex {\headingfont #1}\mark{#1}%
  \vskip 2ex plus 1ex minus 1.5ex}

\newdimen\keyindent

% beginindentedkeys...endindentedkeys - key definitions will be
% indented, but running text, typically used as headings to group
% definitions, will not.
\def\beginindentedkeys{\keyindent=1em}
\def\endindentedkeys{\keyindent=0em}
\endindentedkeys

% paralign - begin paragraph containing an alignment.
% If an \halign is entered while in vertical mode, a parskip is never
% inserted.  Using \paralign instead of \halign solves this problem.
\def\paralign{\vskip\parskip\halign}

% \<...> - surrounds a variable name in a code example
\def\<#1>{{\it #1\/}}

% kbd - argument is characters typed literally.  Like the Texinfo command.
\def\kbd#1{{\tt#1}\null}	%\null so not an abbrev even if period follows

% beginexample...endexample - surrounds literal text, such a code example.
% typeset in a typewriter font with line breaks preserved
\def\beginexample{\par\leavevmode\begingroup
  \obeylines\obeyspaces\parskip0pt\tt}
{\obeyspaces\global\let =\ }
\def\endexample{\endgroup}

% key - definition of a key.
% \key{description of key}{key-name}
% prints the description left-justified, and the key-name in a \kbd
% form near the right margin.
\def\key#1#2{\leavevmode\hbox to \hsize{\vtop
  {\hsize=.75\hsize\rightskip=1em
  \hskip\keyindent\relax#1}\kbd{#2}\hfil}}

\newbox\metaxbox
\setbox\metaxbox\hbox{\kbd{M-x }}
\newdimen\metaxwidth
\metaxwidth=\wd\metaxbox

% metax - definition of a M-x command.
% \metax{description of command}{M-x command-name}
% Tries to justify the beginning of the command name at the same place
% as \key starts the key name.  (The "M-x " sticks out to the left.)
\def\metax#1#2{\leavevmode\hbox to \hsize{\hbox to .75\hsize
  {\hskip\keyindent\relax#1\hfil}%
  \hskip -\metaxwidth minus 1fil
  \kbd{#2}\hfil}}

% threecol - like "key" but with two key names.
% for example, one for doing the action backward, and one for forward.
\def\threecol#1#2#3{\hskip\keyindent\relax#1\hfil&\kbd{#2}\hfil\quad
  &\kbd{#3}\hfil\quad\cr}

%**end of header


\title{GNU Emacsリファレンスカード}

\centerline{(for version \versionemacs)}

\section{Emacsを開始する}

GNU Emacs \versionemacs{}を開始するには、名前を入力する: \kbd{emacs}

\section{Emacsを終了する}

\key{Emacsを中断する(X環境ではアイコン化)}{C-z}
\key{Emacsを完全に終了する}{C-x C-c}

\section{ファイル}

\key{ファイルを{\bf 開く}}{C-x C-f}
\key{ディスク上にファイルを{\bf 保存する}}{C-x C-s}
\key{{\bf 全ての}ファイルを保存する}{C-x s}
\key{バッファに他のバッファの内容を{\bf 挿入する}}{C-x i}
\key{バッファを閉じて別のファイルを開く}{C-x C-v}
\key{バッファを指定したファイルに保存する}{C-x C-w}
\key{バッファのread-onlyステータスを切り替える}{C-x C-q}

\section{ヘルプを使う}

ヘルプシステムは簡単です。 \kbd{C-h}(または\kbd{F1})をタイプして、指示に
従ってください。もし初心者なら、\kbd{C-h t}で{\bf チュートリアル}を読むこと
を薦めます。

\key{ヘルプウィンドウを閉じる}{C-x 1}
\key{ヘルプウィンドウをスクロールする}{C-M-v}

\key{apropos: 指定した文字列にマッチしたコマンドの説明を見る}{C-h a}
\key{特定のキーで動作する関数の説明を見る}{C-h k}
\key{関数の説明を見る}{C-h f}
\key{モード固有の情報を見る}{C-h m}

\section{問題発生からの復旧}

\key{一部だけタイプした状態や実行中のコマンドを{\bf 中断する}}{C-g}
\metax{クラッシュで失ったファイルを{\bf 復旧する}}{M-x recover-session}
\metax{変更を{\bf 取り消す}({\bf アンドゥ})}{C-x u, C-_ {\rm or} C-/}
\metax{バッファを元の内容に復元する}{M-x revert-buffer}
\key{カーソルを中心にウィンドウをスクロールする}{C-l}

\section{インクリメンタルサーチ}

\key{前方を検索する}{C-s}
\key{後方を検索する}{C-r}
\key{正規表現で検索する}{C-M-s}
\key{正規表現で後方を検索する}{C-M-r}

\key{前回の検索文字列を選択する}{M-p}
\key{後の検索文字列を選択する}{M-n}
\key{インクリメンタルサーチを終了する}{RET}
\key{直前の検索文字を取り消す}{DEL}
\key{現在の検索を中断する}{C-g}

同方向の検索を繰り返すには\kbd{C-s}または\kbd{C-r}を複数回タイプしてください。
検索中に\kbd{C-g}をタイプすると、マッチしていない部分のみ取り消されます。

\shortcopyrightnotice

\newcolumn
\section{移動}

\paralign to \hsize{#\tabskip=10pt plus 1 fil&#\tabskip=0pt&#\cr
\threecol{{\bf 移動の単位}}{{\bf 後方}}{{\bf 前方}}
\threecol{文字}{C-b}{C-f}
\threecol{単語}{M-b}{M-f}
\threecol{行}{C-p}{C-n}
\threecol{行頭または行末}{C-a}{C-e}
\threecol{文}{M-a}{M-e}
\threecol{段落}{M-\{}{M-\}}
\threecol{ページ}{C-x [}{C-x ]}
\threecol{S式}{C-M-b}{C-M-f}
\threecol{関数}{C-M-a}{C-M-e}
\threecol{バッファの先頭または末尾}{M-<}{M->}
}

\key{次のスクリーンへスクロールする}{C-v}
\key{前のスクリーンへスクロールする}{M-v}
\key{左へスクロールする}{C-x <}
\key{右へスクロールする}{C-x >}
\key{現在の行を中央/最上部/最下部へスクロールする}{C-l}

\key{指定した行へ移動する}{M-g g}
\key{指定したポイントへ移動する}{M-g c}
\key{現在の行のインデントまで移動する}{M-m}

\section{キル(カット)と削除}

\paralign to \hsize{#\tabskip=10pt plus 1 fil&#\tabskip=0pt&#\cr
\threecol{{\bf キルの単位}}{{\bf 後方}}{{\bf 前方}}
\threecol{文字(キルではなく削除)}{DEL}{C-d}
\threecol{単語}{M-DEL}{M-d}
\threecol{行(終端まで)}{M-0 C-k}{C-k}
\threecol{文}{C-x DEL}{M-k}
\threecol{S式}{M-- C-M-k}{C-M-k}
}

\key{{\bf リージョン(選択範囲)}をキルする}{C-w}
\key{リージョンをキルリングへコピーする}{M-w}
\key{指定した{\it 文字}までキルする}{M-z {\it 文字}}

\key{最後にキルしたものをヤンクする(ペースト)}{C-y}
\key{直前のヤンクを一つ前にキルしたものに置き換える}{M-y}

\section{マーク}

\key{ポイント(カーソル位置)にマークをセットする}{C-@ {\rm or} C-SPC}
\key{ポイントとマークの位置を入れ替える}{C-x C-x}

\key{{\it 引数\/}個先の{\bf 単語}までマークをセットする}{M-@}
\key{{\bf 文}をマークする}{M-h}
\key{{\bf ページ}をマークする}{C-x C-p}
\key{{\bf S式}をマークする}{C-M-@}
\key{{\bf 関数}をマークする}{C-M-h}
\key{{\bf バッファ}全体をマークする}{C-x h}

\section{対話的な置換(Query Replace)}

\key{テキストの文字列を対話的に置換する}{M-\%}
% query-replace-regexp is bound to C-M-% but that can't be typed on
% consoles.
\metax{正規表現を使う}{M-x query-replace-regexp}

対話的な置換において有効なキーは、

\key{現在のものを{\bf 置き換えて}、次に行く}{SPC {\rm or} y}
\key{現在のものを置き換えて、止まる}{,}
\key{置き換えずに次へ{\bf スキップする}}{DEL {\rm or} n}
\key{残りのマッチ全てを置き換える}{!}
\key{一つ前のマッチに{\bf 移動する}}{^}
\key{対話的な置換を{\bf 終了する}}{RET}
\key{再帰編集を開始する(\kbd{C-M-c}で抜ける)}{C-r}

\newcolumn
\section{マルチウィンドウ}

二つのコマンドが書かれているものでは、後者はウィンドウではなくフレームに対する同様の操作
を示しています。

{\setbox0=\hbox{\kbd{0}}\advance\hsize by 0\wd0
\paralign to \hsize{#\tabskip=10pt plus 1 fil&#\tabskip=0pt&#\cr
\threecol{全ての他のウィンドウを閉じる}{C-x 1\ \ \ \ }{C-x 5 1}
\threecol{ウィンドウを上下に分割する}{C-x 2\ \ \ \ }{C-x 5 2}
\threecol{現在のウィンドウを閉じる}{C-x 0\ \ \ \ }{C-x 5 0}
}}
\key{ウィンドウを左右に分割する}{C-x 3}

\key{他のウィンドウをスクロールする}{C-M-v}

{\setbox0=\hbox{\kbd{0}}\advance\hsize by 2\wd0
\paralign to \hsize{#\tabskip=10pt plus 1 fil&#\tabskip=0pt&#\cr
\threecol{別のウィンドウへ切り替える}{C-x o}{C-x 5 o}

\threecol{他のウィンドウでバッファを開く}{C-x 4 b}{C-x 5 b}
\threecol{他のウィンドウでバッファを表示する}{C-x 4 C-o}{C-x 5 C-o}
\threecol{他のウィンドウでファイルを開く}{C-x 4 f}{C-x 5 f}
\threecol{他のウィンドウでread-onlyで開く}{C-x 4 r}{C-x 5 r}
\threecol{他のウィンドウでDiredを開く}{C-x 4 d}{C-x 5 d}
\threecol{他のウィンドウでタグを開く}{C-x 4 .}{C-x 5 .}
}}

\key{ウィンドウを上下に高くする}{C-x ^}
\key{ウィンドウを左右に狭める}{C-x \{}
\key{ウィンドウの左右に広げる}{C-x \}}

\section{整形}

\key{現在の{\bf 行}をインデントする(モード依存)}{TAB}
\key{{\bf リージョン}をインデントする(モード依存)}{C-M-\\}
\key{{\bf S式}をインデントする(モード依存)}{C-M-q}
\key{リージョンを{\it 引数\/}個の列だけインデントする}{C-x TAB}
\key{コメントアウトしてインデントする}{M-;}

\key{ポイントの直後に改行を挿入する}{C-o}
\key{行の後ろを垂直に移動する}{C-M-o}
\key{ポイント周りの空白行を削除する}{C-x C-o}
\key{現在の行を前の行(引数付で次の行)に繋げる}{M-^}
\key{ポイント周りの全ての空白文字を削除する}{M-\\}
\key{ポイント周りに空白を一つだけ置く}{M-SPC}

\key{段落をフィルする(詰め込む)}{M-q}
\key{フィルの幅を{\it 引数}にセットする}{C-x f}
\key{フィルの行頭プレフィックスをセットする}{C-x .}

\key{フェイスをセットする}{M-o}

\section{大文字小文字の変換}

\key{単語を大文字にする(uppercase)}{M-u}
\key{単語を小文字にする(lowercase)}{M-l}
\key{単語の先頭を大文字にする(capitalize)}{M-c}

\key{リージョンを大文字にする}{C-x C-u}
\key{リージョンを小文字にする}{C-x C-l}

\section{ミニバッファ}

以下のキーはミニバッファ内で有効です。

\key{可能な限り補完する}{TAB}
\key{一単語まで補完する}{SPC}
\key{補完して実行する}{RET}
\key{補完を表示する}{?}
\key{以前のミニバッファの入力を使う}{M-p}
\key{以後のミニバッファの入力かデフォルトを使う}{M-n}
\key{正規表現でヒストリを後方に検索する}{M-r}
\key{正規行源でヒストリを前方に検索する}{M-s}
\key{コマンドを中断する}{C-g}

\kbd{C-x ESC ESC}でミニバッファで最後に使ったコマンドを編集し再実行できます。
\kbd{F10}でテキスト端末上でメニューバーを利用できます。

\newcolumn
\title{GNU Emacsリファレンスカード}

\section{バッファ}

\key{他のバッファを開く}{C-x b}
\key{バッファリストを見る}{C-x C-b}
\key{バッファを閉じる}{C-x k}

\section{入れ替え}

\key{{\bf 文字}を入れ替える}{C-t}
\key{{\bf 単語}を入れ替える}{M-t}
\key{{\bf 行}を入れ替える}{C-x C-t}
\key{{\bf S式}を入れ替える}{C-M-t}

\section{スペルチェック}

\key{現在の単語のスペルをチェックする}{M-\$}
\metax{リージョン内の全ての単語をチェックする}{M-x ispell-region}
\metax{バッファ全体のスペルをチェックする}{M-x ispell-buffer}
\metax{on-the-flyスペルチェックを切り替える}{M-x flyspell-mode}

\section{タグ}

\key{タグ(定義)を開く}{M-.}
\metax{新しいタグファイルを指定する}{M-x visit-tags-table}

\metax{全てのファイルを正規表現で検索する}{M-x tags-search}
\metax{全てのファイルに対話的な置換を行う}{M-x tags-query-replace}
\key{前回のタグ検索または対話的な置換を続ける}{M-,}

\section{シェル}

\key{シェルコマンドを実行する}{M-!}
\key{非同期的にシェルコマンドを実行する}{M-\&}
\key{リージョンを入力にシェルコマンドを実行する}{M-|}
\key{リージョンをシェルコマンド出力で置き換える}{C-u M-|}
\key{\kbd{*shell*}ウィンドウでシェルを開始する}{M-x shell}

\section{矩形領域}

\key{矩形領域をレジスタに保存する}{C-x r r}
\key{矩形領域をキルする}{C-x r k}
\key{矩形領域をヤンクする}{C-x r y}
\key{矩形領域を開けてテキストを右にシフトする}{C-x r o}
\key{矩形領域を削除する}{C-x r c}
\key{各行にプレフィックス文字列をつける}{C-x r t}

\section{略称(Abbrevs)}

\key{略称をグローバルに追加する}{C-x a g}
\key{略称をモードローカルに追加する}{C-x a l}
\key{略称に対する展開をグローバルに追加する}{C-x a i g}
\key{略称に対する展開をモードローカルに追加する}{C-x a i l}
\key{略称を展開する}{C-x a e}

\key{一つ前の単語について動的に展開する}{M-/}

\section{その他}

\key{数引数}{C-u {\it 数値}}\break
\key{負引数}{M--}
\key{クォートして挿入する}{C-q {\it 文字}}

\newcolumn
\section{正規表現}

\key{改行以外のあらゆる1文字}{. {\rm(dot)}}
\key{0個以上の繰り返し}{*}
\key{1個以上の繰り返し}{+}
\key{0個か1個の繰り返し}{?}
\key{特殊文字をクォート}{\\}
\key{正規表現の特殊文字{\it c\/}をクォート}{\\{\it c}}
\key{代替(``or'')}{\\|}
\key{グループ化}{\\( {\rm$\ldots$} \\)}
\key{shyなグループ化}{\\(?: {\rm$\ldots$} \\)}
\key{明示的に番号付けしたグループ化}{\\(?NUM: {\rm$\ldots$} \\)}
\key{{\it n\/}番目のグループと同じ文字列}{\\{\it n}}
\key{単語の区切り}{\\b}
\key{単語の区切り以外}{\\B}

\paralign to \hsize{#\tabskip=10pt plus 1 fil&#\tabskip=0pt&#\cr
\threecol{{\bf 単位}}{{\bf マッチ開始}}{{\bf マッチ終了}}
\threecol{行}{^}{\$}
\threecol{単語}{\\<}{\\>}
\threecol{シンボル}{\\_<}{\\_>}
\threecol{バッファ}{\\`}{\\'}
%% FIXME: "`" and "'" isn't displayed correctly in the output PDF file

\threecol{{\bf 文字の種類}}{{\bf これらにマッチ}}{{\bf 他にマッチ}}
\threecol{明示した集合}{[ {\rm$\ldots$} ]}{[^ {\rm$\ldots$} ]}
\threecol{wordシンタックス文字}{\\w}{\\W}
\threecol{シンタックス{\it c}に属する文字}{\\s{\it c}}{\\S{\it c}}
\threecol{カテゴリー{\it c}に属する文字}{\\c{\it c}}{\\C{\it c}}
}

\section{国際文字セット}

\key{主要な言語を指定する}{C-x RET l}
\metax{全てのインプットメソッドを見る}{M-x list-input-methods}
\key{インプットメソッドを有効化/無効化する}{C-\\}
\key{直後のコマンド用に符号化方式をセットする}{C-x RET c}
\metax{全ての符号化方式を見る}{M-x list-coding-systems}
\metax{優先する符号化方式を選択する}{M-x prefer-coding-system}

\section{Info}

\key{Infoドキュメントリーダーを開始する}{C-h i}
\key{指定した関数や変数をInfoから探す}{C-h S}
\beginindentedkeys

ノード内の移動:

\key{前方へスクロールする}{SPC}
\key{後方へスクロールする}{DEL}
\key{ノードの先頭へ移動する}{b}

ノード間の移動:

\key{{\bf 次の}ノード}{n}
\key{{\bf 前の}ノード}{p}
\key{{\bf 上へ}移動する}{u}
\key{名前を指定してメニューアイテムを開く}{m}
\key{数値(1--9)を指定して{\it n\/}番目のメニューアイテムを開く}{{\it n}}
\key{相互参照をたどる(\kbd{l}で戻る)}{f}
\key{最後に見たノードに戻る}{l}
\key{ディレクトリノードに戻る}{d}
\key{Infoファイルのトップノードへ移動する}{t}
\key{名前を指定してノードを移動する}{g}

その他:

\key{Info{\bf チュートリアル}を開く}{h}
\key{索引項目を指定してInfoを引く}{i}
\key{正規表現でノードを検索する}{s}
\key{Infoを{\bf 終了する}}{q}

\endindentedkeys

\newcolumn
\section{レジスタ}

\key{リージョンをレジスタに保存する}{C-x r s}
\key{レジスタの内容をバッファに挿入する}{C-x r i}

\key{ポイントの位置をレジスタに保存する}{C-x r SPC}
\key{レジスタに保存されたポイントへジャンプする}{C-x r j}

\section{キーボードマクロ}

\key{キーボードマクロの定義を{\bf 開始する}}{C-x (}
\key{キーボードマクロの定義を{\bf 終了する}}{C-x )}
\key{最後に定義したキーボードマクロを{\bf 実行する}}{C-x e}
\key{最後のキーボードマクロに定義を追加する}{C-u C-x (}
\metax{キーボードマクロに名前をつける}{M-x name-last-kbd-macro}
\metax{Lispでの定義をバッファに挿入する}{M-x insert-kbd-macro}

\section{Emacs Lispを扱うコマンド}

\key{ポイント直前の{\bf S式}を評価する}{C-x C-e}
\key{現在の{\bf defun}を評価する}{C-M-x}
\metax{{\bf リージョン}を評価する}{M-x eval-region}
\key{ミニバッファで読み取って評価する}{M-:}
\metax{{\bf load-path}上のライブラリをロードする}{M-x load-library}

\section{簡単なカスタマイズ}

\metax{変数やフェイスをカスタマイズする}{M-x customize}

% The intended audience here is the person who wants to make simple
% customizations and knows Lisp syntax.

Emacs Lispでグローバルにキーバインドを設定する(例):

\beginexample%
(global-set-key (kbd "C-c g") 'search-forward)
(global-set-key (kbd "M-\#") 'query-replace-regexp)
\endexample

\section{コマンドを書く}

\beginexample%
(defun \<command-name> (\<args>)
  "\<documentation>" (interactive "\<template>")
  \<body>)
\endexample

例:

\beginexample%
(defun this-line-to-top-of-window (line)
  "Reposition current line to top of window.
With prefix argument LINE, put point on LINE."
  (interactive "P")
  (recenter (if (null line)
                0
              (prefix-numeric-value line))))
\endexample

\kbd{interactive}は、どのような形式で対話的に引数を読み取るかを指定しています。
詳細は\kbd{C-h f interactive RET}で確認してください。

\copyrightnotice

\bye

% Local variables:
% compile-command: "pdftex refcard"
% End:
